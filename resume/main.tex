\documentclass[letterpaper,MMMyyyy,nonstopmode]{simpleresumecv}
\newcommand{\CVAuthor}{Maitreya Venkataswamy}
\newcommand{\CVTitle}{}
\newcommand{\CVNote}{}
\newcommand{\CVWebpage}{}

\hypersetup{
pdftitle={\CVTitle},
pdfauthor={\CVAuthor},
pdfsubject={\CVWebpage},
pdfcreator={XeLaTeX},
pdfproducer={},
pdfkeywords={},
unicode=true,
bookmarks=true,
bookmarksopen=true,
pdfstartview=FitH,
pdfpagelayout=OneColumn,
pdfpagemode=UseOutlines,
hidelinks,
breaklinks}

\newcommand{\Code}[1]{\mbox{\textbf{#1}}}
\newcommand{\CodeCommand}[1]{\mbox{\textbf{\textbackslash{#1}}}}

\begin{document}

\Title{\CVAuthor}

\begin{SubTitle}
%\textbf{US Citizen}
%\par
\url{https://maitreyakv.com}
\,\SubBulletSymbol\,
maitreyakv@gmail.com
\,\SubBulletSymbol\,
+1\,(267)\,273-5586
\href{\CVWebpage}
{\url{\CVWebpage}}
\par
\par
Last Updated: \today
\end{SubTitle}

\begin{Body}

\Section{Work Experience}{Work Experience}{PDF:WorkExperience}

\Entry \textbf{Day Zero Diagnostics -- acquired by BioM\'erieux June 2025}

\Gap

\BulletItem \textit{Senior Data Engineer}
\hfill
\DatestampYMD{2024}{11}{13} -- Present
\SubBulletItem
  Contributed to development of Keynome Cloud, a cloud platform for bacteria species ID and AMR (anti-microbial resistance) profiling from sequenced DNA. Worked primarily on development of backend data processing pipelines in Python deployed on GCP w/Kubernetes, and secondarily on the client frontend in React.
\SubBulletItem
Led development of a CLI program in Rust for uploading sequencing data to our cloud platform. Distributed the tool early-access-program clients and improved ergonomics, performance, and compatibility of the tool using feedback and requirements from clients and partners. 

\Gap

\BulletItem \textit{Data Engineer}
\hfill
\DatestampYMD{2023}{11}{13} -- \DatestampYMD{2024}{11}{13}
\SubBulletItem
Curated MicrohmDB, dataset of sequenced DNA and AMR profiles for 90K+ bacteria isolates, the largest of its kind. Developed ETL pipelines in Python for cleaning and transforming raw sequencing data and AMR test results (from dozens of hospitals, biobanks, public data sources, etc.) into datasets for use in training ML models for predicting AMR profiles from bacteria genetics.

\BigGap

\Entry \textbf{Titan Advanced Energy Solutions}

\Gap

\BulletItem \textit{Data Scientist}
\hfill
\DatestampYMD{2022}{01}{21} -- \DatestampYMD{2023}{10}{26}
\SubBulletItem
Trained machine learning and deep learning models to predict battery State-of-Charge and State-of-Health using ultrasound signal data. Developed signal processing software to engineer robust features from ultrasound data.
\SubBulletItem
Contributed to the development of data pipelines using Prefect, AWS, and Snowflake to collect ultrasound and battery cycling data from multiple long-running data collection campaigns.
\SubBulletItem
Built and maintained validation components of the data pipelines to identify issues with ultrasound captures and send alerts to the battery lab team through Slack. Reduced resolution times for data collection issues from days to hours.
%\SubBulletItem
%Built and maintained internal dashboards using Plotly Dash to monitor and debug data collection campaigns and experiments.

\BigGap

\Entry \textbf{Tagup}

\Gap

\BulletItem \textit{Data Science Intern}
\hfill
\DatestampYMD{2021}{05}{24} -- \DatestampYMD{2021}{08}{13}
\SubBulletItem
Trained Bayesian neural networks to model cooling tower dynamics for model-based reinforcement learning control of HVAC systems. Used TensorFlow Probability for modelling, KubeFlow for distributed training, and Ray RLlib for reinforcement learning.

\BigGap

\Entry \textbf{Brown Center for Computation \& Visualization}

\Gap
  
\BulletItem \textit{Data Science Intern}
\hfill
\DatestampYMD{2020}{10}{01} -- \DatestampYMD{2021}{05}{24}
\SubBulletItem
Processed educational demographic datasets at the undergraduate, Phd, and faculty level to create publication quality figures to support the research of a graduate student in the Sociology Department. Iterated on the visualizations with the graduate student to aid in the investigation of racial trends in the data.


%\Gap
%\begin{Detail}
%\BulletItem
%Relevant Coursework: Machine Learning, Statistical Learning, Deep Learning, Data Engineering
%end{Detail}

\Section
{Skills}
{Skills}
{PDF:Skills}

\Entry \textbf{Technical}
\BulletItem \textit{Advanced}: Python (Prefect, FastAPI, SQLAlchemy, Pandas, Polars, Scikit-Learn, Tensorflow, ...)
\BulletItem \textit{Intermediate}: Rust (Axum, Tokio, Sycamore), SQL (Postgres), Git, AWS, GCP, Docker, Kubernetes
\BulletItem \textit{Beginner}: JavaScript (React), Tailwind CSS

\BigGap

\Entry \textbf{Soft}
\BulletItem Requirements gathering, Collaborative development, Learning domain knowledge

\Section
{Education}
{Education}
{PDF:Education}

\Entry \textbf{Brown University}
\hfill
\DatestampYMD{2020}{09}{01} -- \DatestampYMD{2021}{10}{01}
\BulletItem M.S. Data Science

\Entry \textbf{Georgia Institute of Technology}
\hfill
\DatestampYMD{2016}{08}{19} -- \DatestampYMD{2020}{05}{01}
\BulletItem B.S. Aerospace Engineering, Minor in Computer Science

%\Section
%{Research Experience}
%{Research Experience}
%{PDF:ResearchExperience}
%
%\Entry
%\textbf{Visual Prosthesis Project},
%Brown University
%\hfill
%\DatestampYMD{2020}{10}{01} -- \DatestampYMD{2021}{05}{24}
%\begin{Detail}
%\BulletItem
%Collected data from the prosthesis' depth camera to build a probabalistic model of the sensor noise. Simulated the depth camera in an artificial environment described using tetrahedral meshes. Used the simulation to develop and test image segmentation, obstacle detection, and path planning algorithms
%%\BulletItem
%%Focus: Programming, Path Planning, Obstacle Detection
%\end{Detail}
%
%\Entry
%\textbf{Computational Combustion Laboratory},
%Georgia Institute of Technology
%\hfill
%\DatestampYMD{2017}{01}{01} -- \DatestampYMD{2020}{05}{01}
%\begin{Detail}
%\BulletItem
%Setup, ran, and post-processed stratified fluid dynamics simulations using the lab's proprietary incompressible fluid dynamics solver. Programmed post-processing utilities for data manipulation, data visualization, and generation of publication quality plots. Performed statistical analysis of turbulence properties. Utilized DoD HPC supercomputers to perform simulations.
%%\BulletItem
%%Created a GUI using Python and TKinter for simulation setup and configuration for the lab's proprietary multi-phase, reacting compressible flow solver. GUI was designed to check for errors in the user's simulation inputs and to report the errors effectively to the user.
%%\BulletItem
%%Focus: Programming, Post-Processing, Data Visualization, Simulation
%\end{Detail}

%\Section
%{Highlighted Academic Projects}
%{Highlighted Academic Projects}
%{PDF:Highlighted Academic Projects}
%
%\Entry
%\textbf{Temporally-Aware Deep Learning for Video Frame Classification}
%\begin{Detail}
%\BulletItem
%Used different temporally-aware deep learning architectures (2D-CNN, 3D-CNN, Conv-LSTM) to classify the frames of the videos of the Rat Social Interaction dataset. Compared the performance vs. complexity of each architecture to demonstrate the importance of being able to capture spatio-temporal patterns using the proper neural network architecture.
%\end{Detail}
%
%\Entry
%\textbf{Deep Learning for Cassava Leaf Disease Classification}
%\begin{Detail}
%\BulletItem
%Trained a convolutional neural network in Tensorflow to classify diseases in images of Cassava leafs. Utilized transfer learning of EfficientNet and image augmentation to improve the classifier performance.
%\end{Detail}

%\Entry
%\textbf{Simulation of 2D Mobile Robot Localization Using a Particle Filter}
%\begin{Detail}
%\BulletItem
%Simulated the localization process of a 2D mobile robot using a particle filter and rangefinder sensors. Wrote an accompanying online article to introduce the concept of Bayes Filter and Particle Filter to an unfamiliar audience.
%\end{Detail}

% \Entry
% \textbf{Development of a Machine Learning Model for NMR Data}, 3rd place in ML4SCI 2020 Hackathon
% \begin{Detail}
% \BulletItem
% Developed a multi-target regression model for predicting quantum numbers from simulated NMR data using a combination of polynomial regression and gradient boosted trees.
% \end{Detail}

% \Entry
% \textbf{Imitation Learning for Cart-Pole Systems}
% \begin{Detail}
% \BulletItem
% Trained supervised machine learning models (gradient boosted trees, adaptive boosted trees, random forest, polynomial regression) to imitate a Differential Dynamic Programming controller to perform the "swing-up" maneuver of a simulated cart-pole system.
% \end{Detail}

% \Entry
% \textbf{Predicting Wind Power Generation Using Weather Forecast Simulations}
% \begin{Detail}
% \BulletItem
% Used weather simulation data from the Global Forecast System to train machine learning models to predict the wind power output of the Pacific Northwest. Created a cloud application to incrementally download weather simulation data and store it in a MongoDB database. Created a web application to showcase the project.
% \end{Detail}

%\Entry
%\textbf{Simulated System Control Using Differential Dynamic Programming}
%\begin{Detail}
%\BulletItem
%Implemeted the Differential Dynamic Programming algorithm in MATLAB for use with arbitrary systems and used it to simulate the "swing-up" of a cart-pole system.
%\end{Detail}

%\Entry
%\textbf{Design of CubeSat Mission for Titan Imaging}, Capstone project for Bachelor's Degree
%\begin{Detail}
%\BulletItem
%Performed high level systems engineering work for the development of a conceptual satellite mission to Titan. Worked on a team of six to develop a full proposal for the mission and the satellite. Created simulations of low-thrust orbit maneuvers using Simulink, and performed power system analysis to size the power systems.
%\end{Detail}

%\Entry
%\textbf{Simulation of Gas Turbine and Ramjet Engines}
%\begin{Detail}
%\BulletItem
%Created a component level simulation tool in Matlab to simulate the operation of gas turbine and ramjet engine components. Utilized the simulation framework to optimize engine parameters (bypass ratio, fuel ratio, etc.) to achieve optimal performance (specific thrust, thrust specific fuel consumption, etc.) for different flight conditions.
%\end{Detail}

%\Entry
%\textbf{Parallel Simulation of Vortex Interaction}
%\begin{Detail}
%\BulletItem
%Simulation of 2D systems of large number of vortices using adaptive time-stepping Euler integration. Implemented in C using OpenMP for parallelization.
%\end{Detail}

%\Entry
%\textbf{Simulation of Multi-Predator Prey Population Dynamics}
%\begin{Detail}
%\BulletItem
%Simulation of systems of multiple predators and prey using cellular automata implemented in Python, with analytic comparisons to Lotka-Volterra population dynamics model.
%\end{Detail}

%\Entry
%\textbf{Simulation of Road Traffic using Cellular Automata}
%\begin{Detail}
%\BulletItem
%Simulation of road traffic in Atlanta using cellular automata simulation implemented in C++ using OpenMP for parallelization
%\end{Detail}

%\Section
%{Publications}
%{Publications}
%{PDF:Publications}
%
%\SubSection
%{Conferences}
%{Conferences}
%{PDF:Conferences}
%
%% Declare a new group to limit the scope of \MaxNumberedItem to this subsection.
%\begingroup
%\renewcommand{\MaxNumberedItem}{[8888]}
%
%\Gap
%\NumberedItem{[1]}
%{R. Ranjan, \underline{M. Venkataswamy}, and S. Menon,
%``Investigation of effects equation of state and differential diffusion on fully developed stratified turbulent channel flow,''
%in \textit{Bulletin of the American Physical Society},
%Atlanta, Georgia, USA,
%\DatestampYM{2018}{011}.}
%
%\SubSection
%{Journals}
%{Journals}
%{PDF:Journals}
%
%% Declare a new group to limit the scope of \MaxNumberedItem to this subsection.
%\begingroup
%\renewcommand{\MaxNumberedItem}{[8888]}
%
%\NumberedItem{[1]}
%{Ranjan, R., \underline{Venkataswamy, M.}, & Menon, S. (2020). "Dynamic one-equation-based subgrid model for large-eddy simulation of stratified turbulent flows", Phys. Rev. Fluids, 5, 064601.}
%
%\endgroup

%\Section
%{Awards \&\newline
%Scholarships}
%{Awards \& Scholarships}
%{PDF:AwardsAndScholarships}

%\textbf{PURA Travel Award},
%Georgia Institute of Technology
%\hfill
%\DatestampY{2018}
%\begin{Detail}
%\Item
%Awarded for presenting research at a professional conference.
%\end{Detail}

%\textbf{Dean's List},
%Georgia Institute of Technology
%\hfill
%\DatestampY{2016} -- \DatestampY{2020}
%\begin{Detail}
%\Item
%Awarded each semester for ataining a semester GPA of at least 3.00.
%\end{Detail}

\end{Body}

\end{document}
